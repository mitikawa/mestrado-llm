\section{Resultados}
\label{sec:resultados}

Nesta seção, apresentamos os resultados experimentais divididos por arquitetura. Iniciamos com o modelo base (BERT) para estabelecer uma linha de base, seguido pelas variações de arquitetura (destilada e otimizada). Cada subseção detalha o desempenho nas tarefas de Detecção de Subjetividade (SD) e MGTD, concluindo com a análise de viés e robustez.

\subsection{Arquitetura A: BERT-Base (Baseline)}
\label{ssec:bertbase}

Nesta configuração, utilizamos o modelo \texttt{bert-base-uncased}...

\subsubsection{Desempenho nas Tarefas Base}

As Tabelas \ref{tab:bert_sd} e \ref{tab:bert_mgtd} detalham a performance obtida.

% Tabela SD - BERT
\begin{table}[h]
\centering
\small
\caption{Resultados do BERT-Base na Detecção de Subjetividade (NewsSD).}
\label{tab:bert_sd}
\begin{tabular}{lcccc}
\toprule
\textbf{Classe} & \textbf{Precision} & \textbf{Recall} & \textbf{F1-Score} & \textbf{Support} \\
\midrule
Objetivo (OBJ)    & 0.00 & 0.00 & 0.00 & 0 \\
Subjetivo (SUBJ)  & 0.00 & 0.00 & 0.00 & 0 \\
\midrule
\textit{Accuracy}     &      &      & 0.00 & 0 \\
\textit{Macro Avg}    & 0.00 & 0.00 & 0.00 & 0 \\
\textit{Weighted Avg} & 0.00 & 0.00 & 0.00 & 0 \\
\bottomrule
\end{tabular}
\end{table}

% Tabela MGTD - BERT
\begin{table}[h]
\centering
\small
\caption{Resultados do BERT-Base em MGTD (SemEval-2024).}
\label{tab:bert_mgtd}
\begin{tabular}{lcccc}
\toprule
\textbf{Classe} & \textbf{Precision} & \textbf{Recall} & \textbf{F1-Score} & \textbf{Support} \\
\midrule
Humano  & 0.00 & 0.00 & 0.00 & 0 \\
Máquina & 0.00 & 0.00 & 0.00 & 0 \\
\midrule
\textit{Accuracy}     &      &      & 0.00 & 0 \\
\textit{Macro Avg}    & 0.00 & 0.00 & 0.00 & 0 \\
\textit{Weighted Avg} & 0.00 & 0.00 & 0.00 & 0 \\
\bottomrule
\end{tabular}
\end{table}

\subsubsection{Análise de Viés e Robustez}

A seguir, apresentamos a análise cruzada utilizando o índice de subjetividade gerado pelo BERT-Base.

\paragraph{Viés de Estilo (Q1):} 

\paragraph{Robustez à Subjetividade (Q2):} 


\subsection{Arquitetura B: DistilBERT}
\label{ssec:distilbert}

Nesta configuração, utilizamos o modelo \texttt{distilbert-base-uncased} ...

\subsubsection{Desempenho nas Tarefas Base}

As Tabelas \ref{tab:distilbert_sd} e \ref{tab:distilbert_mgtd} apresentam os relatórios de classificação detalhados.

% Tabela SD - DistilBERT
\begin{table}[h]
\centering
\small
\caption{Resultados do DistilBERT na Detecção de Subjetividade (NewsSD).}
\label{tab:distilbert_sd}
\begin{tabular}{lcccc}
\toprule
\textbf{Classe} & \textbf{Precision} & \textbf{Recall} & \textbf{F1-Score} & \textbf{Support} \\
\midrule
Objetivo (OBJ)    & 0.00 & 0.00 & 0.00 & 0 \\
Subjetivo (SUBJ)  & 0.00 & 0.00 & 0.00 & 0 \\
\midrule
\textit{Accuracy}     &      &      & 0.00 & 0 \\
\textit{Macro Avg}    & 0.00 & 0.00 & 0.00 & 0 \\
\textit{Weighted Avg} & 0.00 & 0.00 & 0.00 & 0 \\
\bottomrule
\end{tabular}
\end{table}

% Tabela MGTD - DistilBERT
\begin{table}[h]
\centering
\small
\caption{Resultados do DistilBERT em MGTD (SemEval-2024).}
\label{tab:distilbert_mgtd}
\begin{tabular}{lcccc}
\toprule
\textbf{Classe} & \textbf{Precision} & \textbf{Recall} & \textbf{F1-Score} & \textbf{Support} \\
\midrule
Humano  & 0.00 & 0.00 & 0.00 & 0 \\
Máquina & 0.00 & 0.00 & 0.00 & 0 \\
\midrule
\textit{Accuracy}     &      &      & 0.00 & 0 \\
\textit{Macro Avg}    & 0.00 & 0.00 & 0.00 & 0 \\
\textit{Weighted Avg} & 0.00 & 0.00 & 0.00 & 0 \\
\bottomrule
\end{tabular}
\end{table}

\subsubsection{Análise de Viés e Robustez}

\paragraph{Viés de Estilo (Q1):} 


\paragraph{Robustez à Subjetividade (Q2):} 


\subsection{Arquitetura C: RoBERTa}
\label{ssec:roberta}

Finalmente, avaliamos o modelo \texttt{roberta-base},...

\subsubsection{Desempenho nas Tarefas Base}

As Tabelas \ref{tab:roberta_sd} e \ref{tab:roberta_mgtd} mostram os resultados.

% Tabela SD - RoBERTa
\begin{table}[h]
\centering
\small
\caption{Resultados do RoBERTa na Detecção de Subjetividade (NewsSD).}
\label{tab:roberta_sd}
\begin{tabular}{lcccc}
\toprule
\textbf{Classe} & \textbf{Precision} & \textbf{Recall} & \textbf{F1-Score} & \textbf{Support} \\
\midrule
Objetivo (OBJ)    & 0.00 & 0.00 & 0.00 & 0 \\
Subjetivo (SUBJ)  & 0.00 & 0.00 & 0.00 & 0 \\
\midrule
\textit{Accuracy}     &      &      & 0.00 & 0 \\
\textit{Macro Avg}    & 0.00 & 0.00 & 0.00 & 0 \\
\textit{Weighted Avg} & 0.00 & 0.00 & 0.00 & 0 \\
\bottomrule
\end{tabular}
\end{table}

% Tabela MGTD - RoBERTa
\begin{table}[h]
\centering
\small
\caption{Resultados do RoBERTa em MGTD (SemEval-2024).}
\label{tab:roberta_mgtd}
\begin{tabular}{lcccc}
\toprule
\textbf{Classe} & \textbf{Precision} & \textbf{Recall} & \textbf{F1-Score} & \textbf{Support} \\
\midrule
Humano  & 0.00 & 0.00 & 0.00 & 0 \\
Máquina & 0.00 & 0.00 & 0.00 & 0 \\
\midrule
\textit{Accuracy}     &      &      & 0.00 & 0 \\
\textit{Macro Avg}    & 0.00 & 0.00 & 0.00 & 0 \\
\textit{Weighted Avg} & 0.00 & 0.00 & 0.00 & 0 \\
\bottomrule
\end{tabular}
\end{table}

\subsubsection{Análise de Viés e Robustez}

\paragraph{Viés de Estilo (Q1):} 

\paragraph{Robustez à Subjetividade (Q2):} 