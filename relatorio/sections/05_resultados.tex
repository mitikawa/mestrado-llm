\section{Resultados}
\label{sec:resultados}

Nesta seção, apresentamos os resultados experimentais divididos por arquitetura. Iniciamos com o modelo base (BERT) para estabelecer uma linha de base, seguido pelas variações de arquitetura (destilada e otimizada). Cada subseção detalha o desempenho nas tarefas de Detecção de Subjetividade (SD) e MGTD, concluindo com a análise de viés e robustez.

\subsection{Arquitetura A: BERT-Base (Baseline)}
\label{ssec:bertbase}

Nesta configuração, utilizamos o modelo \textit{bert-base-uncased}.

\subsubsection{Desempenho nas Tarefas Base}

As Tabelas \ref{tab:bert_sd} e \ref{tab:bert_mgtd} detalham a performance obtida.

% Tabela SD - BERT
\begin{table}[h]
\centering
\small
\caption{Resultados do BERT-Base na Tarefa 1 (dados de teste).}
\label{tab:bert_sd}
\begin{tabular}{lcccc}
\toprule
\textbf{Classe} & \textbf{Precisão} & \textbf{Recall} & \textbf{F1-Score} & \textbf{Suporte} \\
\midrule
OBJ  & 0.90 & 0.74 & 0.81 & 215 \\
SUBJ & 0.55 & 0.80 & 0.65 & 85 \\
\midrule
\textit{Acurácia}     &      &      & 0.76 & 300 \\
\textit{Média Ar.}    & 0.73 & 0.77 & 0.73 & 300 \\
\textit{Média Ge.} & 0.80 & 0.76 & 0.77 & 300 \\
\bottomrule
\end{tabular}
\end{table}

% Tabela MGTD - BER
\begin{table}[h]
\centering
\small
\caption{Resultados do BERT-Base na Tarefa 2 (dados de teste).}
\label{tab:bert_mgtd}
\begin{tabular}{lcccc}
\toprule
\textbf{Classe} & \textbf{Precisão} & \textbf{Recall} & \textbf{F1-Score} & \textbf{Suporte} \\
\midrule
Humano  & 0.95 & 0.43 & 0.59 & 16272 \\
Máquina & 0.66 & 0.98 & 0.79 & 18000 \\
\midrule
\textit{Acurácia}     &      &      & 0.72 & 34272 \\
\textit{Média Ar.}    & 0.80 & 0.71 & 0.69 & 34272 \\
\textit{Média Ge.} & 0.80 & 0.72 & 0.69 & 34272 \\
\bottomrule
\end{tabular}
\end{table}

\subsubsection{Análise de Viés e Robustez}

A seguir, apresentamos a análise cruzada utilizando o índice de subjetividade gerado pelo BERT-Base.

\paragraph{Viés de Estilo (Q1):} A Tabela \ref{tab:bert_bias} mostra a comparação do $I_{subj}$ para dados de teste. A Figura \ref{fig:bert_kde} mostra a distribuição visualmente. Percebe-se a diferença estatística clara.

\begin{table}[h]
\centering
\small
\caption{Índice de Subjetividade Médio ($I_{subj}$) calculado pelo BERT-Base.}
\label{tab:bert_bias}
\begin{tabular}{lcc}
\toprule
\textbf{Origem} & \textbf{Média $I_{subj}$} & \textbf{Desvio Padrão} \\
\midrule
Texto Humano    & 0.2874 & 0.1356 \\
Texto Máquina   & 0.1784 & 0.1204 \\
\midrule
\textit{Diferença} & \multicolumn{2}{c}{$p_{valor} = 0.00$} \\
\bottomrule
\end{tabular}
\end{table}

\begin{figure}
    \centering
    \includegraphics[width=1\linewidth]{figures/Fig1_Distribuicao_Subjetividade.png}
    \caption{Curva de estimativa de densidade de Kernel.}
    \label{fig:bert_kde}
\end{figure}

\paragraph{Robustez à Subjetividade (Q2):}
Para investigar se o detector utiliza a subjetividade como um forte indicador para a classificação, analisamos a correlação entre o índice $I_{subj}$ e a probabilidade de classe positiva (IA). A Figura \ref{fig:bert_corr} apresenta uma regressão logística sobre os dados de teste. A linha de tendência indica que, \textbf{em média}, textos na região de alta objetividade ($I_{subj} < 0.2$) recebem probabilidades elevadas de serem sintéticos. Por outro lado, à medida que o texto incorpora elementos subjetivos, a probabilidade média do detector na classe IA decai continuamente. 

\begin{figure}[h]
    \centering
    \includegraphics[width=1\linewidth]{figures/Fig2_Correlacao_Binned.png}
    \caption{Correlação entre o Índice de Subjetividade e a Probabilidade de Detecção (Classe IA). A linha vermelha indica o ajuste logístico.}
    \label{fig:bert_corr}
\end{figure}

A análise de sensibilidade por faixas é apresentada na Figura \ref{fig:bert_perf}. O desempenho do modelo não é uniforme: enquanto o detector atinge métricas altas (F1-Score $> 0.90$) para textos "Muito Objetivos", observa-se uma degradação consistente de performance conforme a subjetividade aumenta. Isso indica que a robustez do modelo é condicional ao estilo do texto.

\begin{figure}[h]
    \centering
    \includegraphics[width=1\linewidth]{figures/Fig3_Desempenho_por_Faixa.png}
    \caption{Desempenho do classificador (Acurácia e F1-Score) segmentado por quartis de subjetividade.}
    \label{fig:bert_perf}
\end{figure}

No entanto, essa degradação global esconde uma assimetria de robustez entre as classes, exposta nas Figuras \ref{fig:bert_ai} e \ref{fig:bert_human}. Observa-se que a taxa de acerto para textos de IA (Recall) permanece próxima de $99\%$ independentemente do nível de subjetividade (Fig. \ref{fig:bert_ai}), indicando que o modelo é extremamente eficaz em capturar artefatos de máquina. 

A queda na performance global (Fig. \ref{fig:bert_perf}) explica-se pela distribuição das classes: a região de alta subjetividade é composta majoritariamente por textos humanos. Como o modelo possui baixa especificidade nesta região — acertando apenas entre $30\%$ a $50\%$ dos humanos objetivos (Fig. \ref{fig:bert_human}) — o peso desses falsos positivos domina a métrica global.

\begin{figure}[h]
    \centering
    \begin{minipage}{0.48\textwidth}
        \centering
        \includegraphics[width=\linewidth]{figures/Fig5_Performance_IA_Recall.png}
        \caption{Taxa de Acerto em IA (Recall).}
        \label{fig:bert_ai}
    \end{minipage}\hfill
    \begin{minipage}{0.48\textwidth}
        \centering
        \includegraphics[width=\linewidth]{figures/Fig6_Performance_Humano_Recall.png}
        \caption{Taxa de Acerto em Humanos.}
        \label{fig:bert_human}
    \end{minipage}
\end{figure}

Por fim, a Figura \ref{fig:bert_boxplot} mostra o o \textit{boxplot} da distribuição de subjetividade nos acertos e erros do modelo. Observa-se que os Verdadeiros Negativos (TN - Humanos Corretos) possuem a maior média de subjetividade, indicando que a presença de marcas estilísticas pessoais é um fator de proteção contra falsos positivos. Em contrapartida, os Falsos Positivos (FP - Humano $\to$ IA) apresentam uma distribuição deslocada para a esquerda em relação aos TN, sugerindo que humanos com escrita excessivamente técnica entram na zona de risco de classificação.

Entretanto, é notável a presença de numerosos \textit{outliers} de alta subjetividade na classe Verdadeiros Positivos (TP). Isso demonstra que, embora o detector tenha um forte viés para associar objetividade à IA (evidenciado pela concentração da caixa à esquerda), ele retém capacidade de detecção em casos onde a IA gera textos subjetivos.

\begin{figure}[h]
    \centering
    \includegraphics[width=1\linewidth]{figures/Fig4_Analise_Erro_Boxplot.png}
    \caption{Distribuição de subjetividade por resultado da classificação.}
    \label{fig:bert_boxplot}
\end{figure}


\subsection{Arquitetura B: DistilBERT}
\label{ssec:distilbert}

Nesta configuração, utilizamos o modelo \texttt{distilbert-base-uncased} ...

\subsubsection{Desempenho nas Tarefas Base}

As Tabelas \ref{tab:distilbert_sd} e \ref{tab:distilbert_mgtd} apresentam os relatórios de classificação detalhados.

% Tabela SD - DistilBERT
\begin{table}[h]
\centering
\small
\caption{Resultados do DistilBERT na Detecção de Subjetividade (NewsSD).}
\label{tab:distilbert_sd}
\begin{tabular}{lcccc}
\toprule
\textbf{Classe} & \textbf{Precisão} & \textbf{Recall} & \textbf{F1-Score} & \textbf{Suporte} \\
\midrule
Objetivo (OBJ)    & 0.00 & 0.00 & 0.00 & 0 \\
Subjetivo (SUBJ)  & 0.00 & 0.00 & 0.00 & 0 \\
\midrule
\textit{Acurácia}     &      &      & 0.00 & 0 \\
\textit{Média Ar.}    & 0.00 & 0.00 & 0.00 & 0 \\
\textit{Média Ge.} & 0.00 & 0.00 & 0.00 & 0 \\
\bottomrule
\end{tabular}
\end{table}

% Tabela MGTD - DistilBERT
\begin{table}[h]
\centering
\small
\caption{Resultados do DistilBERT em MGTD (SemEval-2024).}
\label{tab:distilbert_mgtd}
\begin{tabular}{lcccc}
\toprule
\textbf{Classe} & \textbf{Precisão} & \textbf{Recall} & \textbf{F1-Score} & \textbf{Suporte} \\
\midrule
Humano  & 0.00 & 0.00 & 0.00 & 0 \\
Máquina & 0.00 & 0.00 & 0.00 & 0 \\
\midrule
\textit{Acurácia}     &      &      & 0.00 & 0 \\
\textit{Média Ar.}    & 0.00 & 0.00 & 0.00 & 0 \\
\textit{Média Ge.} & 0.00 & 0.00 & 0.00 & 0 \\
\bottomrule
\end{tabular}
\end{table}

\subsubsection{Análise de Viés e Robustez}

\paragraph{Viés de Estilo (Q1):} 


\paragraph{Robustez à Subjetividade (Q2):} 


\subsection{Arquitetura C: RoBERTa}
\label{ssec:roberta}

Finalmente, avaliamos o modelo \texttt{roberta-base},...

\subsubsection{Desempenho nas Tarefas Base}

As Tabelas \ref{tab:roberta_sd} e \ref{tab:roberta_mgtd} mostram os resultados.

% Tabela SD - RoBERTa
\begin{table}[h]
\centering
\small
\caption{Resultados do RoBERTa na Detecção de Subjetividade (NewsSD).}
\label{tab:roberta_sd}
\begin{tabular}{lcccc}
\toprule
\textbf{Classe} & \textbf{Precisão} & \textbf{Recall} & \textbf{F1-Score} & \textbf{Suporte} \\
\midrule
Objetivo (OBJ)    & 0.00 & 0.00 & 0.00 & 0 \\
Subjetivo (SUBJ)  & 0.00 & 0.00 & 0.00 & 0 \\
\midrule
\textit{Acurácia}     &      &      & 0.00 & 0 \\
\textit{Média Ar.}    & 0.00 & 0.00 & 0.00 & 0 \\
\textit{Média Ge.} & 0.00 & 0.00 & 0.00 & 0 \\
\bottomrule
\end{tabular}
\end{table}

% Tabela MGTD - RoBERTa
\begin{table}[h]
\centering
\small
\caption{Resultados do RoBERTa em MGTD (SemEval-2024).}
\label{tab:roberta_mgtd}
\begin{tabular}{lcccc}
\toprule
\textbf{Classe} & \textbf{Precisão} & \textbf{Recall} & \textbf{F1-Score} & \textbf{Suporte} \\
\midrule
Humano  & 0.00 & 0.00 & 0.00 & 0 \\
Máquina & 0.00 & 0.00 & 0.00 & 0 \\
\midrule
\textit{Acurácia}     &      &      & 0.00 & 0 \\
\textit{Média Ar.}    & 0.00 & 0.00 & 0.00 & 0 \\
\textit{Média Ge.} & 0.00 & 0.00 & 0.00 & 0 \\
\bottomrule
\end{tabular}
\end{table}

\subsubsection{Análise de Viés e Robustez}

\paragraph{Viés de Estilo (Q1):} 

\paragraph{Robustez à Subjetividade (Q2):} 