\section{Considerações Éticas}
\label{sec:consideracoes_eticas}

A pesquisa em MGTD carrega implicações éticas, dado o potencial de aplicação destas tecnologias em ambientes sensíveis como a educação e o jornalismo.

\paragraph{Impacto Social:} Como destacado na introdução, o efeito danoso na integridade acadêmica é um motivador para o desenvolvimento de detectores \cite{wang-etal-2024-semeval-2024}. No entanto, os resultados preliminares sobre o viés de subjetividade trazem um outro risco: a possibilidade de discriminação estilística. Se os detectores tendem a classificar textos mais objetivos como sendo gerados por IA, indivíduos que naturalmente adotam um estilo de escrita mais neutro, técnico ou informacionalmente denso podem ser desproporcionalmente alvo de falsas acusações. A implementação de tais ferramentas sem a devida calibração para essa variância pode resultar em injustiças.

\paragraph{Risco de Uso Maldoso:} Pesquisas como as nossas que expõem as fraquezas dos detectores atuais (especificamente a sua vulnerabilidade a textos subjetivos) pode ser utilizada de forma maldosa. Agentes mal-intencionados podem utilizar estas conclusões para instruir LLMs a adotarem posturas mais subjetivas ou emocionais deliberadamente para confudir modelos de detecção e disseminar desinformação automatizada (como fake news). Contudo, defendemos que a transparência sobre estas vulnerabilidades é essencial para o desenvolvimento de sistemas de defesa mais robustos e semanticamente conscientes.

\paragraph{Uso de Dados e Energia:} Este estudo utilizou estritamente \textit{datasets} públicos e anonimizados que não contêm informações sensíveis, mitigando riscos de privacidade. Além disso, ao optarmos pelo \textit{fine-tuning} de modelos pré-treinados (BERT, RoBERTa) em vez do uso de LLMs mais custosas, buscamos minimizar a pegada de carbono associada ao custo computacional da pesquisa.