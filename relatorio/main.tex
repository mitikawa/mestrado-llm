\documentclass[11pt]{article}
\usepackage{acl}

% Pacotes essenciais
\usepackage[T1]{fontenc}
\usepackage[utf8]{inputenc}
\usepackage[portuguese]{babel} 
\usepackage{dirtytalk}

\usepackage{times}
\usepackage{latexsym}
\usepackage{graphicx} 
\usepackage{booktabs}
\usepackage{amsmath}   
\usepackage{float}

\title{Análise da Interseção entre Subjetividade Linguística e Detecção de Texto Gerado por IA}

% Os 3 autores do grupo
\author{João Guilherme de Freitas Rocha \\
  UFF \\
  \texttt{\small{jofreitas@id.uff.br}} \\\And
  Marcelo Campanelli\\
  UFF \\
  \texttt{\small{marcelocampanelli@id.uff.br}} \\\And
  Mateus Abreu Itikawa \\
  UFF \\
  \texttt{\small{mitikawa@id.uff.br}} \\}

\begin{document}
\maketitle


\begin{abstract}
A crescente capacidade dos \textit{Large Language Models} (LLMs) de produzirem textos fluentes e coerentes tornou a distinção entre escrita humana e sintética um desafio técnico relevante. Nesse contexto, detectores de texto gerado por IA (\textit{Machine Generated Text Detection} — MGTD) ainda apresentam fragilidades, sobretudo na generalização para modelos não vistos ou para domínios fora de distribuição. Paralelamente, estudos recentes sugerem que LLMs exibem um viés estilístico sistemático em direção à objetividade, ao passo que a escrita humana tende a expressar maior subjetividade. 

Motivado por essa hipótese, este trabalho investiga empiricamente a relação entre subjetividade linguística e a eficiência de detectores MGTD. Para isso, em diferentes configurações de arquitetura BERT-Based (BERT, DistilBERT,DeBERTa e RoBERTa), treinamos dois classificadores especializados: um modelo de Detecção de Subjetividade (SD) e um modelo de MGTD, aplicando o primeiro sobre os dados de teste do segundo. A análise cruzada revelou padrões robustos: (i) textos humanos são significativamente mais subjetivos do que textos gerados por LLMs; (ii) detectores MGTD apresentam forte dependência dessa característica estilística; (iii) níveis extremos de subjetividade em textos humanos aumentam a taxa de falsos positivos, enquanto maior objetividade favorece classificações incorretas como IA.

Os resultados indicam que parte substancial do desempenho dos detectores deriva da exploração indireta do grau de subjetividade. Concluímos que a subjetividade tende a atuar como variável crítica para a robustez dos detectores e sugerimos que futuras linhas de pesquisa desenvolvam técnicas de calibração capazes de dissociar estilo e origem textual, mitigando penalizações injustas a textos humanos muito opinativos ou criativos.
\end{abstract}


\section{Introdução}
\label{sec:introducao}

Os Large Language Models (LLMs) em muito expandiram a fronteira da geração de texto em linguagem natural. Esses modelos demonstram uma capacidade de produzir conteúdo fluente, coeso e contextualmente relevante, sendo cada vez mais difícil distingui-los da escrita humana em diversos cenários \cite{wang-etal-2024-semeval-2024}. Embora essa tecnologia ofereça avanços significativos, ela também apresenta riscos substanciais, como a disseminação automatizada de desinformação, a erosão da integridade acadêmica e a automação de fraudes.

Em resposta a esses riscos, a Detecção de Texto Gerado por Máquina (Machine Generated Text Detection - MGTD) emergiu como um relevante campo de pesquisa. Os métodos atuais de MGTD, embora eficazes para geradores conhecidos, demonstram uma fragilidade significativa em robustez, falhando em generalizar para modelos não vistos ou domínios fora de distribuição (OOD) \cite{yang2023surveydetectionllmsgeneratedcontent, kuznetsov-etal-2024-robust}. Muitos detectores SOTA (State-of-the-Art) parecem depender de artefatos estatísticos específicos do gerador ou do domínio de treinamento, em vez de características semânticas ou estilísticas fundamentais.

Em paralelo, a literatura emergente sugere que os LLMs podem possuir um "viés estilístico" intrínseco. Estudos recentes \cite{Reinhart_2025, Mu_oz_Ortiz_2024} indicam que os LLMs, especialmente aqueles ajustados por instrução (instruction-tuned), tendem a produzir textos "informacionalmente densos" que favorecem uma linguagem mais objetiva em detrimento da expressão subjetiva, que é uma característica da autoria humana \cite{pang-lee-2004-sentimental}. 

Assim, a análise de sentimentos, mais especificamente a Detecção de Subjetividade (Subjectivity Detection - SD) surge como uma possível ferramenta auxiliar na MGTD. Como mostrado em \cite{chaturvedi}, a SD é um tema em pesquisa há algumas décadas, começando por modelos com "features" criadas manualmente, como Naive-Bayes e árvores, passando por aprendizagem automático de features e por modelos baseados em vetores de palavras neurais. Já a literatura mais recente \cite{song2025largelanguagemodelssubjective} mostra como "LLMs são mais adaptadas a modelar julgamentos sútis como humanos".

Este trabalho propõe-se a investigar essa lacuna. O objetivo não é apenas avaliar a efetividade de detectores, mas principalmente analisar como a característica linguística da subjetividade impacta a eficácia da detecção. Para tal, formulamos duas questões centrais:
\begin{itemize}
    \item \textbf{Q1:} Existe uma diferença estatisticamente significativa na propensão à subjetividade/objetividade entre textos gerados por humanos e por LLMs?
    \item \textbf{Q2:} O desempenho dos classificadores MGTD é influenciado pelo nível de subjetividade de um texto? (i.e., textos objetivos são mais fáceis ou difíceis de detectar como sendo de IA?)
\end{itemize}

Para responder a estas questões, adotamos uma metodologia de análise controlada. Treinamos dois classificadores especializados usando a mesma arquitetura base (BERT-base) com adaptadores LoRA (Low-Rank Adaptation): um Modelo de Classificação de Subjetividade ($M_{SD}$) treinado no corpus NewsSD-ENG \cite{antici2024corpussentencelevelsubjectivitydetection}, e um Modelo de Detecção MGTD ($M_{MGTD}$) treinado nos dados da SemEval-2024 Task 8 \cite{wang-etal-2024-semeval-2024}. Subsequentemente, utilizamos o $M_{SD}$ para dissecar o desempenho do $M_{MGTD}$, avaliando sua robustez em estratificações de texto objetivas e subjetivas.

As contribuições deste trabalho são uma análise quantitativa da correlação entre subjetividade e autoria de LLMs e uma avaliação da robustez de detectores MGTD padrão em relação a este viés estilístico.

O restante deste artigo está estruturado da seguinte forma: A Seção \ref{sec:trabalhos_relacionados} revisa a literatura em SD e MGTD. A Seção \ref{sec:analise_datasets} detalha os corpora utilizados. A Seção \ref{sec:metodo} apresenta nossa arquitetura unificada e a metodologia de análise. A Seção \ref{sec:resultados} apresenta os resultados experimentais. Finalmente, as seções \ref{sec:conclusao}, \ref{sec:limitacoes} e \ref{sec:consideracoes_eticas} discutem as conclusões, limitações e implicações éticas do estudo.
\section{Trabalhos Relacionados}
\label{sec:trabalhos_relacionados}

Esta seção avalia o estado da arte acadêmico em duas áreas centrais do Processamento de Linguagem Natural (Natural Language Processing - NLP): Detecção de Subjetividade (SD) e Detecção de Texto Gerado por Máquina (MGTD). Embora tradicionalmente investigadas como campos de estudo separados, objetivamos avaliar se a análise de características estilísticas como subjetividade pode auxiliar no âmbito de MGTD.

\subsection{Detecção de Subjetividade}
\label{ssec:fundamentos_cs}

A primeira tarefa envolve a distinção entre sentenças que apresentam fatos (objetivas) e aquelas que expressam opiniões (subjetivas). \citet{pang-lee-2004-sentimental} é um trabalho seminal que introduziu o "Cornell Subjectivity Dataset v1.0 (SUBJ)". O dataset consiste em 5000 sentenças extraídas de resumos de enredo do IMDb (consideradas objetivas) e 5000 sentenças extraídas de críticas de usuários do Rotten Tomatoes (consideradas subjetivas). A pesquisa subsequente reconheceu as limitações deste dataset, onde o domínio (enredo vs. crítica) não é um substituto perfeito para a propriedade linguística da subjetividade. A literatura moderna moveu-se em direção a anotações explícitas.

Introduzido por \citet{bjerva2020subjqadatasetsubjectivityreview}, o dataset \textbf{SubjQA} foca na subjetividade dentro do contexto de Question Answering (QA). O dataset \textbf{NewsSD-ENG} \cite{antici2024corpussentencelevelsubjectivitydetection} tem como principal contribuição o desenvolvimento de "novas diretrizes de anotação que não se limitam a pistas específicas do idioma" e que podem ser aplicadas a qualquer língua.

\subsection{Detecção de Texto Gerado por Máquina}
\label{ssec:panorama_mgtd}

O rápido avanço dos LLMs trouxe atenção urgente para a tarefa de MGTD \cite{Fu2025DetectAnyLLM} para mitigar o uso indevido de LLMs. Vários *surveys* recentes mapearam o estado da arte em MGTD \cite{yang2023surveydetectionllmsgeneratedcontent, wu2024surveyllmgeneratedtextdetection}. Estas revisões categorizam as abordagens SOTA da seguinte forma:
\begin{itemize}
    \item \textbf{Detecção White-Box vs. Black-Box:} Baseada no acesso do detector ao modelo de origem.
    \item \textbf{Métodos Zero-Shot:} Detectores que não requerem treinamento em dados específicos de MGTD (ex: baseados em perplexidade).
    \item \textbf{Métodos Baseados em Fine-Tuning:} A abordagem mais comum, onde um modelo (ex: RoBERTa) é ajustado em uma tarefa de classificação binária (humano vs. máquina).
\end{itemize}

O desafio unificador em MGTD é a robustez. Os detectores SOTA atuais são notavelmente frágeis e falham em cenários Out-of-Distribution (OOD) \cite{wu2024surveyllmgeneratedtextdetection}, pertubações adversariais, espaçamento randômico ou contra simples parafraseamento \cite{he2024mgtbenchbenchmarkingmachinegeneratedtext}.

O benchmark da tarefaSemEval-2024 Task 8 \cite{wang-etal-2024-semeval-2024} foi explicitamente projetado para testar a robustez dos detectores em cenários complexos (multidomínio, multigerador e multilíngue), dividida em subtarefas de classificação binária (A), detecção de fonte (B) e detecção de ponto de mudança (C). A análise dos sistemas vencedores na SemEval-2024 Task 8 revela a tendência dominante: "Para todas as subtarefas, os melhores sistemas usaram LLMs" \cite{wang-etal-2024-semeval-2024}.

\subsection{Interseção do Problema: Viés Estilístico de LLMs}
\label{ssec:intersecao}

Esta seção conecta as Tarefas 1 e 2, sintetizando a literatura que analisa as características estilísticas da saída do LLM e como essas características impactam a detecção. A literatura recente começou a quantificar as diferenças estilísticas entre a escrita humana e a gerada por LLM, e um padrão claro está emergindo: LLMs exibem um viés mensurável em direção à objetividade.

\citet{Mu_oz_Ortiz_2024} descobriu que "As saídas dos LLM usam mais números, símbolos e auxiliares (sugerindo linguagem objetiva) do que textos humanos". Em contraste, "Humanos tendem a exibir emoções negativas mais fortes". \citet{Reinhart_2025} mostraram que modelos "instruction-tuned", por serem treinados a responder perguntas e resolver problemas, tendem a ter um estilo de escrita distinto com tendência a nomes e informacionalmente denso, mesmo quando requisitados a manter um estilo de fala e escrita informal. \cite{opara2025distinguishingaigeneratedhumanwrittentext} propôs um framework que integra análise "estilométrica" com teorias psicolinguísticas.

Este viés estilístico tem implicações diretas para a robustez do detector MGTD. Se os detectores são treinados em grandes volumes de texto de IA que compartilham esse viés "objetivo", eles podem falhar ao encontrar texto de IA que imita a subjetividade humana. Ademais, a falha em domínios OOD é um problema complexo. \citet{kuznetsov-etal-2024-robust} focaram sua pesquisa exatamente nisso, avaliando a robustez e a capacidade de detectores performarem bem em geradores ou domínios semânticos não vistos.
\section{Análise das Bases de Dados}
\label{sec:analise_datasets}

Esta secção detalha os dois \textit{datasets} utilizados, um para cada tarefa, e articula o desafio metodológico imposto pela sua diferença de granularidade.

\subsection{Tarefa 1}
\label{ssec:dataset_sd}

Para a tarefa 1, o \textit{dataset} é o \textbf{NewsSD-ENG} \cite{antici2024corpussentencelevelsubjectivitydetection}. Este é um corpus focado em notícias de língua inglesa que abordam tópicos controversos. O NewsSD-ENG fornece anotação explícita em nível de sentença.

O \textit{dataset} é composto por 1.049 sentenças, com as estatísticas de partição detalhadas na Tabela \ref{tab:stats_newssd}.

\begin{table}[h]
\centering
\small
\begin{tabular}{@{}lcccc@{}}
\toprule
\textbf{Partição} & \textbf{\# Artigos} & \textbf{\# Sentenças} & \textbf{OBJ} & \textbf{SUBJ} \\
\midrule
Treino & 16 & 731 & 487 (12) & 244 (46) \\
Dev & 3 & 99 & 45 (3) & 54 (8) \\
Teste & 4 & 219 & 106 (4) & 113 (16) \\
\midrule
\textbf{Total} & \textbf{23} & \textbf{1.049} & \textbf{638 (19)} & \textbf{411 (70)} \\
\bottomrule
\end{tabular}
\caption{Estatísticas do corpus NewsSD-ENG, adaptado de \citet{antici2024corpussentencelevelsubjectivitydetection}. Em parênteses o número de sentenças disputadas (aquelas em que dois anotadores não concordaram e um terceiro fez a anotação).}
\label{tab:stats_newssd}
\end{table}

A análise da Tabela \ref{tab:stats_newssd} revela duas observações metodologicamente críticas que informam o nosso desenho experimental (Secção \ref{sec:metodo}):
\begin{enumerate}
    \item \textbf{Tamanho do Dataset:} O conjunto de treino é notavelmente reduzido (n=731). Esta escassez de dados impõe um risco significativo de \textit{overfitting} para LLMs submetidos a \textit{fine-tuning} completo.
    \item \textbf{Desbalanceamento de Classes:} O conjunto de treino exibe um desbalanceamento considerável (proporção de aprox. 2:1) em favor da classe objetiva (\texttt{OBJ}).
\end{enumerate}

Ambos os fatores justificam a adoção de uma abordagem de \textit{Parameter-Efficient Fine-Tuning} (PEFT), como o LoRA, e a utilização de estratégias de mitigação de desbalanceamento, como a ponderação de classes na função de perda (\textit{weighted loss}).

\subsection{Tarefa 2}
\label{ssec:dataset_mgtd}

Para a tarefa de MGTD, utilizamos os dados da \textbf{SemEval-2024 Task 8} \cite{wang-etal-2024-semeval-2024}. Em linha com o escopo do nosso trabalho, focamo-nos na \textbf{Subtarefa A} (classificação binária) e na trilha \textbf{monolíngue em inglês}.

Os dados são fornecidos em formato JSONL, divididos nas partições de treino e desenvolvimento (\textit{dev}). Cada instância no \textit{dataset} é composta pelo texto integral (\texttt{text}), o rótulo de origem (\texttt{label}, indicando \textit{human} ou \textit{machine}), o modelo gerador (ex: \texttt{GPT-4}, \texttt{Llama}, etc.), a fonte (\texttt{source}) e um identificador (\texttt{id}).

\subsection{Desafio de Granularidade dos Dados}
\label{ssec:granularidade}

Uma observação fundamental é a disparidade na granularidade dos dados entre as duas tarefas. A Tarefa 1 (SD) opera em \textbf{nível de sentença}. Em contrapartida, a Tarefa 2 (MGTD) opera em \textbf{nível de documento}, onde cada instância (\texttt{text}) consiste em múltiplos parágrafos.

Esta discrepância dificulta uma comparação direta e constitui um desafio metodológico central. Aplicar diretamente o classificador $M_{MGTD}$ (treinado em documentos) a uma sentença, ou aplicar o $M_{SD}$ (treinado em sentenças) diretamente a um documento $D_{MGTD}$, levaria a uma perda de performance.

Para superar este desafio, a nossa metodologia (detalhada na Secção \ref{sec:metodo}) emprega uma \textbf{estratégia de agregação}, onde o classificador $M_{CS}$ é aplicado iterativamente às sentenças de um documento $D_{MGTD}$ para calcular um "Índice de Subjetividade" agregado.
\section{Metodologia}
\label{sec:metodo}

Placeholder metodo
\section{Resultados}
\label{sec:resultados}

Placeholder resultados
\section{Conclusão}
\label{sec:conclusao}

Placeholder conclusao
\section{Limitações}
\label{sec:limitacoes}

Apesar das contribuições apresentadas na análise da interseção entre subjetividade e autoria, este estudo opera sob restrições metodológicas e de dados que delimitam o escopo de suas conclusões.

\paragraph{Restrições do Dataset de Subjetividade:} A limitação mais crítica reside na dimensão e especificidade do corpus NewsSD-ENG. Com apenas 731 sentenças de treino e um desbalanceamento de classes de 2:1, a capacidade de generalização dos modelos $M_{SD}$ é restrita. Ademais, o NewsSD-ENG é focado estritamente no domínio jornalístico. Em contraste, o dataset de MGTD (SemEval-2024) é multi-domínio, abrangendo desde escrita criativa até técnica. Existe, portanto, um risco de \textit{domain shift}, onde o modelo de subjetividade pode interpretar incorretamente características estilísticas de outros domínios (como ficção) como sendo subjetividade jornalística.

\paragraph{Forma de Agregação:} A estratégia de calcular o Índice de Subjetividade ($I_{subj}$) através da média aritmética das probabilidades das sentenças  constitui uma simplificação. Esta abordagem ignora a estrutura discursiva e a interdependência sequencial do texto. Um documento pode ser globalmente objetivo, mas conter uma única sentença altamente subjetiva que altera sua percepção semântica, nuance esta que pode focar diluída em uma média simples.

\paragraph{Propagação de Erro:} A metodologia adota uma abordagem em cascata (\textit{pipeline}), onde a análise das Questões de Pesquisa (Q1 e Q2) depende inteiramente da precisão do classificador $M_{SD}$. Erros de classificação no primeiro estágio (falsos positivos de subjetividade) propagam-se para a análise subsequente, podendo introduzir ruído nas correlações observadas entre subjetividade e detecção de autoria.

\paragraph{Escopo Arquitetural:} Finalmente, focamos em modelos baseados em arquiteturas \textit{encoder-only} da família BERT (DistilBERT, BERT e RoBERTa). Embora eficientes, estes modelos possuem uma janela de contexto e capacidade de raciocínio inferiores aos \textit{Large Language Models} (LLMs) generativos atuais (e.g., GPT-4, Llama-3). É possível que modelos maiores capturem nuances de subjetividade pragmática que escapam às arquiteturas avaliadas neste estudo, porém questões limitantes de recursos definiram as escolhas.

\paragraph{Otimização de Hiperparâmetros:} Devido a restrições temporais, não foi possível realizar uma busca de hiperparâmetros (como \textit{Grid Search}). Adotamos valores padrões onde possível para parâmetros críticos como \textit{learning rate}, \textit{batch size} e configurações do LoRA (rank e alpha). É provavel que o desempenho dos modelos poderia ser superior com um ajuste fino mais rigoroso.
\section{Considerações Éticas}
\label{sec:consideracoes_eticas}

A pesquisa em MGTD carrega implicações éticas, dado o potencial de aplicação destas tecnologias em ambientes sensíveis como a educação e o jornalismo.

\paragraph{Impacto Social:} Como destacado na introdução, o efeito danoso na integridade acadêmica é um motivador para o desenvolvimento de detectores \cite{wang-etal-2024-semeval-2024}. No entanto, os resultados preliminares sobre o viés de subjetividade trazem um outro risco: a possibilidade de discriminação estilística. Se os detectores tendem a classificar textos mais objetivos como sendo gerados por IA, indivíduos que naturalmente adotam um estilo de escrita mais neutro, técnico ou informacionalmente denso podem ser desproporcionalmente alvo de falsas acusações. A implementação de tais ferramentas sem a devida calibração para essa variância pode resultar em injustiças.

\paragraph{Risco de Uso Maldoso:} Pesquisas como as nossas que expõem as fraquezas dos detectores atuais (especificamente a sua vulnerabilidade a textos subjetivos) pode ser utilizada de forma maldosa. Agentes mal-intencionados podem utilizar estas conclusões para instruir LLMs a adotarem posturas mais subjetivas ou emocionais deliberadamente para confudir modelos de detecção e disseminar desinformação automatizada (como fake news). Contudo, defendemos que a transparência sobre estas vulnerabilidades é essencial para o desenvolvimento de sistemas de defesa mais robustos e semanticamente conscientes.

\paragraph{Uso de Dados e Energia:} Este estudo utilizou estritamente \textit{datasets} públicos e anonimizados que não contêm informações sensíveis, mitigando riscos de privacidade. Além disso, ao optarmos pelo \textit{fine-tuning} de modelos pré-treinados (BERT, RoBERTa) em vez do uso de LLMs mais custosas, buscamos minimizar a pegada de carbono associada ao custo computacional da pesquisa.

%\bibliographystyle{acl_natbib} 
\bibliography{custom}

\end{document}